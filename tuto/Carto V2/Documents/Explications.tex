Ex1

Découvrir à la fois python et le svg, format graphique pour le web.

Le svg a besoin d'un en-tête précis. 
Comme il est assez long, on va concaténer plusieurs lignes avec le $+$ de python.
Différences entre simples et doubles quotes (guillemets/apostrophes)

Toujours svg: Attention à la boîte de visualisation, avec son B Majuscule.
Il n'y a pas d'échelle, juste des coordonnées avec l'axe positif des y qui descend (celui des x est normal)
le point (0,0) est en haut à gauche de la fenêtre (ou de l'écran)

viewBox="x1 y1 largeur hauteur" 

On donne des noms français à nos objets et variables pour ne pas les confondre avec des fonctions python

Un polygone svg peut avoir beaucoup d'attributs: contour, couleur, etc.
Le terme polygone est mal choisi car ce peut être une somme de polygones: Z
D'où le nom de variable polycheminsvg

En théorie, chaque objet a un identifiant unique. 
En pratique, les navigateurs sont tolérants.
Mais l'idéal est de décrire les différentes composantes connexes d'un même objet de la façon suivante:
<objet id="Obj-1" première comp connexe></objet>
<objet id="Obj-2" seconde comp connexe></objet>
Et on fera en sorte que tous les Obj-* aient la même couleur.
Cf. éclaté de la région parisienne

On découvre les fonctions replace et write de python

On écrit du texte en svg

Essai possible avec anaconda-navigator et JupyterLab

Ex2

Comme Ex1, mais  on ajoute un cadre, des cercles, un calque

Le cadre est inséré dans un calque: <g...> </g>: plus propre, plus adapté à des CSS

Ensuite: python: 1ere ouverture d'un fichier (cercles), découverte  de csv.reader, des listes et des boucles

On fabrique les cercles «\,à la main\,»

Jusque là, moitié python, moitié svg

Ex3: protection contre les bugs

Des variables récupèrent directement le contenu d'une ligne-liste
On fait une pause d'une seconde
On ajoute le rayon à une liste
Et si on ne fait pas attention, ce rayon est pris comme un caractère et non comme un nombre
De ce fait, on découvre que 10 est plus petit que 5 (ordre alphanumérique)

Le min est une fonction de base de python
la bibliothèque time est nécessaire pour la fonction sleep

La fonction exit() peut être utile aussi pour arrêter le programme, mais la syntaxe qui la suit doit être correcte (différence avec le __END__ de Perl)

Ex4

Retour au svg

Initiation aux feuilles de style, qui offrent des attributs identiques aux objets qui y font référence: directement ou via un méta-objet
Ici la classe cercle servira de référence

Pour python: reprise de l'Ex2 avec une nuance: découverte de la fonction format

Ce sera le moyen d'introduire beautiful soup et json

Ex5 

Oubli temporaire du svg

Programme: lire les 565 fichiers des résultats du 1er tour des législatives
et les synthétiser dans un seul fichier. On s'intéresse aux taux d'abstention supérieurs à la «\,norme\,», choisie par l'expérimentateur (ex: 30\%).

On va donc produire un fichier du type ResultatPourAbstentions30pc (ou ...32pc, etc.) dans les lignes duquel on écrira 
l'ident de la circonscription, le total des inscrits, le pourcentage d'abstentions et ce que serait le pourcentage réel d'un parti des Abstentionnistes (pour le calcul, cf. le fichier partiAbstention.pdf)

Il y a des histoires complexes d'identifiant des circonscriptions

Ensuite, on repère les 2 lignes qui nous intéressent (découverte du startswith), on sépare les valeurs obtenues par des tabulations, on fait attention aux indentations propres à python, et on écrit le résultat dans le fichier ResultatPourAbstentions30pc

Preuve qu'on peut aisément travailler sur beaucoup de fichiers à la fois

Ex6
Calcul de quelques indicateurs du fichier obtenu

Facile: on fabrique une liste à partir d'une colonne choisie,
on calcule les quantiles (import numpy), le min, le max, l'écart-type, etc.
Cela nous servira pour définir des seuils de couleurs 

Ex7 et 8
Problèmes avec polygones (paths)

Fabrication d'une première carte avec un jeu de coordonnées







